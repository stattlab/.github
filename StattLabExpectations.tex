\documentclass{article}
\usepackage[margin=0.8in]{geometry}
\usepackage{hyperref}
\setlength{\parindent}{0em}
\setlength{\parskip}{0.5em}
\title{Common Lab Member Expectations}
\author{Antonia Statt}
\begin{document}
\maketitle

\section*{Mission Statement}
Welcome to the Statt Lab! The Statt lab has three central goals:

\begin{itemize}
	\item To do quality science and disseminate our progress to the scientific and broader community 
	\item To develop each lab member to become a successful scientist
	\item To maintain a collegial and intellectually stimulating environment
\end{itemize}

I expect you to actively participate and contribute to the lab mission while you are part of the  group.
Your reasons for wanting to get a graduate degree will affect how we tailor your experience here to best meet your overall goals, so let me know what you’d like to accomplish while you’re here. From my side, I see your goals as many-fold: thinking independently and critically, conducting sound research, presenting at scientific meetings and group meetings, participate in outreach and informal science education, and submitting your completed work as manuscript(s) for publication in peer-reviewed journals.

Agreements, obligations, and expectations are often implicit and sometimes students don't have a  clear idea of what they can expect when joining a research lab, so I have compiled this document to give you some information about  that. 
Much of it might be common knowledge, but writing down common expectations helps prevent misunderstandings and it also gives you a tool to asses your own progress.
\textbf{Please review this entire document carefully. }

I want your  degree and research experience to be a fulfilling process. 
At the end of it, you will have completed a substantial body of research that will make a contribution to your field and gained teaching and/or mentoring experience. In my experience three things are crucial for completing your degree:  independence, perseverance, and a positive attitude.

The main part of my role is to act as a mentor and provide direction and advice on your research. I will try to provide assistance with science, project design, data analysis, writing, and giving informal and formal presentations of your work; however, you are responsible for your own your project and the goal is to graduate as independent researcher. 
 
 
\section*{What you can expect from me}
\begin{itemize}
	\item I will set the broad scientific direction for the lab and seek out the means to pursue those directions. This will include helping you to find a research topic, writing grants to fund our research. Additionally, I will seek out collaborators for our work to further your opportunities.  This includes but is not limited to: writing and re-writing grants, meeting funding agency demands,  finding and nurturing collaborations with experimental groups here at UIUC and elsewhere, establish and maintain relationships with companies, mediating  existing collaborations, deciding on research directions, responding to external opportunities.  These activities consume most of my time and might not be immediately noticable to you as a graduate student. Some of these external factors will also most likely have an appreciable effect on your research journey, which we will navigate together.
	\item I am committed to mentoring you now and in the future. I am committed to your education and training while in my lab, and to advising and guiding your career development. I will work to promote you and your work, and I am your advocate.  I will provide honest letters of evaluation for you
	when you request them. I will work tirelessly for the good of the lab group; the success of every member of our group is my top priority, no matter their personal strengths and weaknesses, or career goals.
	\item I will be available for regular scheduled meetings and will provide timely review of research and other relevant items. I will do my best to provide an open door policy and respond quickly to e-mails. Please be aware that there will be times when I will be unavailable due to other obligations. For abstracts and small questions, I will generally be able to review in quickly, for proposals and papers we should agree on a timeline with internally set deadlines  beforehand.
	\item 	I will provide a work environment that is intellectually stimulating, supportive, safe, and free from harassment. I will strive to be supportive, equitable, accessible, encouraging, and respectful. I am open to your suggestions on how to improve your experience in the lab.  I will try my best to understand your unique situation, and mentor you accordingly. I am mindful that each student comes from a different background and has different professional goals.
	\item I will encourage you to attend and present at scientific meetings and conferences. These meetings are important to showcase your work and for the networking opportunities as you pursue positions after your time in my lab ends. They are also important for me for finding and nurturing collaborations, developing new ideas and maintaining professional relationships. At meetings and conferences I will actively support you by promoting your work, introduce you where appropriate and help you establish a professional network.  I will discuss data ownership and authorship policies regarding papers and presentations with you. 
	\item  I will lead by example and facilitate your training in complementary skills needed to be a
	successful scientist, such as oral and written communication skills, grant writing, lab management,
	mentoring, and scientific professionalism. I will strongly encourage you to seek opportunities
	in teaching, even if not required for your degree program. I will also  encourage you to
	gain practice in mentoring undergraduate students.
\end{itemize}


\section*{What I expect from you} 


\textbf{You will take ownership of your research and educational experience}
\begin{itemize}
	\item Acknowledge that you have the primary responsibility for the successful completion of
	your degree. This includes commitment to your research and classwork. You
	should maintain a high level of professionalism, self-motivation, engagement, scientific curiosity,
	and ethical standards.  As long as you are
	meeting expectations, you can largely set your own schedule. I expect you to work a minimum of about half of the time during core businesses hours to promote and facilitate interactions with other members of the lab. It is your responsibility to talk with me if you are having difficulty completing your research. 
	\item Be knowledgeable of the policies, deadlines, and requirements of the graduate program,
	the graduate school, and the university. Comply with all institutional policies, including academic
	program milestones, ethical conduct, laboratory practices, and rules related to safety.
	\item Ensure that you meet regularly with me and provide me with updates on the progress and
	results of all your activities, be proactive and timely. Set weekly goals using the "SMART" framework (specific, measurable, attainable, relevant, time-bound). When possible, goals should be constrained to span under 1-2 days and spread out over the week (e.g. not all goals due on the same day).
	Make sure that you also use this meeting time to communicate
	new ideas that you have about your work and challenges that you are facing. You should be sure that I am aware of your key objectives, research questions, and basic approach at all times. This includes information about how mentorship with an undergraduate student might be going, classwork, and professional development activities.
	I cannot address or advise about issues that you do not bring to my attention.
	\item Take full advantage of resources for education and professional development by the department, college, and university. This includes making connections to other groups and professors, as those connections might turn out to be quite important in your future career (i.e. letters of recommendation, future colleagues, connection to different industries, and so on ).
\end{itemize}


\textbf{You will be a team player}
\begin{itemize}
		\item Strive to be the very best group member. Take part in shared lab responsibilities and use 
	resources carefully and mindfully. Maintain a safe office space. Be respectful, tolerant of, and work collegially	with all lab colleagues: respect individual differences in values, personalities, work styles,
	and theoretical perspectives. Respect your lab colleagues possessions and privacy. Be respectful of shared spaces, consider noise and cleanliness levels.
	\item Attend and actively participate in all group meetings and seminars that are part of your
	educational program. Participation in group meetings does not mean only presenting your own
	work, but providing support to others in the lab through shared insight and questions. Do your part to create a climate of engagement and mutual respect. Ideally, you should be asking questions during each group meeting.
	\item Be a good collaborator. Engage in collaborations within and beyond our lab group. Collaborations
	are more than just publishing papers together. They demand effective and frequent communication,
	mutual respect, trust, and shared goals. Effective collaboration is an extremely important
	component of the success of our lab.
	\item  Make it a priority to help other lab members with their projects and mentor/train other students. This includes adding to and improving the shared group wiki: \href{http://statt.web.illinois.edu/wiki/doku.php?id=start}{http://statt.web.illinois.edu/wiki/doku.php?id=start}.
   \item Acknowledge the efforts of collaborators and give credit. This includes other members of the lab as well as
those outside the lab.
\item When attending outreach or recruiting events, conferences, or professional meetings please remember that you are an ambassador for yourself, your research field, and the whole group. Anything you do does reflect on all of us and we all should put our best foot forward when interacting with each other and with the larger community.
\end{itemize}

\textbf{You will develop strong research skills}
\begin{itemize}
\item Write down your ideas in proposal form, with clear research questions and objectives. These informal proposals can be the basis of dialog with me, your committee,  or co-advisors and are starting points for fellowship applications and a thesis proposal.
\item  Challenge yourself by presenting your work at (group) meetings and seminars  and
by preparing scientific articles that effectively present your work to others in the field.
\item Substantial progress and contributions to the research field is required. You will be expected to be lead author on at least two journal papers submissions, preferably more by the time you graduate.  
\item Keep up with the literature so that you can have a hand in guiding your own research. Block
at least one hour per week to peruse literature searches and read at least one paper per week, significantly more at the beginning of a project. Document your literature search, keep notes.
\item  Maintain detailed, organized, and accurate notes. Be aware that your notes,
records and all tangible research data are lab property, therefore they should be in English. When you leave the lab, I
encourage you to take copies of your data and notes with you. One full set of all data must stay in the lab,
with appropriate and accessible documentation. Regularly backup your  data.
\item  Be responsive to advice and constructive criticism. The feedback you get from me, your colleagues,
your committee members, and your course instructors is intended to improve your scientific work.
\end{itemize} 

\textbf{You will communicate clearly}
\begin{itemize}
\item Remember that all of us are ``new'' at various points in our careers. If you feel uncertain, overwhelmed,
or want additional support, please overtly ask for it. I welcome these conversations and
view them as necessary.
\item  Clearly communicate about your schedule and workload at different times. Do not cancel meetings with me if you feel that you have not made adequate progress on your research; these might be the most critical times to
meet.
\item  Let me know the style of communication that will work best for you.  If there is
something about my mentoring style that is proving difficult for you, please tell me so that we have an opportunity to find an approach that works for both of us and supports the progress towards your degree. No single style works for everyone; no one style is expected to work all the time. Realize that the advising style you might need during a specific stage of your research career might not be necessarily the `easiest' style and is up to my discretion.
\item  Respond promptly to emails from anyone in our lab group and show up on time and prepared for meetings. 
\item  Discuss policies on work hours, sick leave and vacation with me directly. Remember that your graduate studies are a serious full-time job and requires appropriate commitment of time and effort. 
\item  Discuss policies on authorship and attendance at professional meetings with me before
beginning any projects to ensure that we are in agreement. I expect you to submit relevant
research results in a timely manner. Barring unusual circumstances, it is my policy that students
are first-author on all work for which they took the lead on data generation and preparation of the
initial draft of the manuscript.
\item Discuss any outgoing communication to potential collaborators beforehand and cc me on any important email communication to third parties, even if my direct involvement is not necessary.
\item Strive to meet internal and external deadlines. They are important to manage your progress, and I expect you to work your best to maintain the goals. If you haven't accomplished the goals we set, you need to explain the steps you took towards it, the challenges that arose, and how you'll overcome them.
\end{itemize}

\section*{Yearly evaluation}

Each year we will have an evaluation – this will help us to determine things that are going well and identify areas for improvement. I will tell you if I am satisfied with your progress and help identify steps you can take to fix any concerns. This is also an opportunity for you to communicate to me what I can do to help you succeed. Tell me if you feel that you need more guidance, more independence, to meet more or less often, etc. Remember that
I am your advocate, as well as your advisor.  At that time, we will also work on an individual development plan and self assessment.
	
All of this at once might sound overwhelming, you do not need to perfect every single point in this document right away. For every new student we also expect a learning curve and I acknowledge that everyone is coming from different backgrounds with various experiences. This document is intended to be helping you identify your strengths and areas of growth so that you can grow into the successful graduate student you want to be. Below is a list of very concrete things you can start doing and thinking about:
\begin{itemize}
	\item  Ask yourself what the motivation is for completing a phD and what you would like to do with it once you graduate. Being a graduate student is hard work and can feel difficult at times, so having a very clear and strong vision of why you want this can be extremely helpful to rely on. If helpful for you, write your motivation and goals down.
	\item Read the wiki: \href{http://statt.web.illinois.edu/wiki/doku.php?id=start}{http://statt.web.illinois.edu/wiki/doku.php?id=start}, especially the \href{http://statt.web.illinois.edu/wiki/doku.php?id=guidelines:general}{general guidelines}.
	\item Ask which papers and books to read.
	\item Think about how you want to organize your notes (meeting notes, technical notes, and notes about papers -- you will read and write a lot).
	\item Start doing Bash, Python, git and LaTeX tutorials.
	\item Plan which classes to take and when, ask around. Familiarize yourself with professional development opportunities available to you.
	\item Attend all individual and group meetings. Be on time and prepared, ask questions. 
	\item Actively participate in recruiting and outreach events. 
	\item Think about big and small milestones you would like to achieve and when. Those goals can be bigger things like paper publications or conference talks and posters,  but also group presentations, complete classwork, mastering a skill, finishing a code implementation, setting up and maintaining a professional online presence, and so on.  Some of your milestones will be very common (publish papers) but others might be very specific for you and what you want to achieve in the end. Write them down and regularly update the list -- they will be helpful for an individual development plan and self assessment.
\end{itemize}

\end{document}
